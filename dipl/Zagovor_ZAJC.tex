%\documentclass[handout]{beamer}
\documentclass{beamer}

\usetheme{Madrid}
\usecolortheme{whale}
\useinnertheme{circles} 
\useoutertheme{split}
\setbeamertemplate{blocks}[rounded][shadow=true]

\usepackage[slovene]{babel}
\usepackage[OT2,T1]{fontenc}
\usepackage[utf8]{inputenc}
\usepackage{lmodern}
\usepackage{amsfonts,amsmath,amssymb,amsthm}
\usepackage{colortbl}
\usepackage[all]{xy}


\newtheorem{izrek}[theorem]{Izrek}
\newtheorem{trditev}[theorem]{Trditev}
\newtheorem{posledica}[theorem]{Posledica}
\newtheorem{vprasanje}[theorem]{Vpra\v anje}
\newtheorem{lema}[theorem]{Lema}
\newtheorem{definicija}{Definicija}

%\beamertemplatenavigationsymbolsempty

\title{Enakostranični trikotniki in Jordanove krivulje}
\author[Nejc Zajc]{Nejc Zajc \\ mentor: izr.~prof.~dr.~Aleš Vavpetič}
%\institute{University of Ljubljana, Slovenia}
\date{Zagovor dela diplomskega seminarja, 19.\ 4.\ 2022}


%nova definicija ukaza
\newcommand{\ind}[3][]{\text{ind}_{#1}(#2, #3)}

\begin{document}
%%%%%%%%%%%%%%%%%%%%%%%%%%%%%%%%%%%%%%%%
\begin{frame}
\maketitle

\end{frame}
%%%%%%%%%%%%%%%%%%%%%%%%%%%%%%%%%%%%%%%%
\begin{frame}

\frametitle{Obstoj}
\pause
\begin{izrek}
Vsaka Jordanova krivulja v ravnini vsebuje oglišča enakostraničnega trikotnika.
\end{izrek}

\end{frame}
%%%%%%%%%%%%%%%%%%%%%%%%%%%%%%%%%%%%%%%%
\begin{frame}

\frametitle{Nova pojma}
\pause
\begin{itemize}
\item Trioda
\pause
\item Presečno število preslikav
\end{itemize}

\end{frame}
%%%%%%%%%%%%%%%%%%%%%%%%%%%%%%%%%%%%%%%%
\begin{frame}

\frametitle{Ugotovitve}
\begin{izrek}\label{izr:glavni}
Trioda $T$  v ravnini vsebuje oglišča enakostraničnega trikotnika, ki ima za eno izmed oglišč končno točko triode.
\end{izrek}
\pause
\begin{posledica}
Vse razen kvečjemu dveh točk Jordanove krivulje $J$ v ravnini so ogliščne točke krivulje $J$.
\end{posledica}

\end{frame}
%%%%%%%%%%%%%%%%%%%%%%%%%%%%%%%%%%%%%%%%
\begin {frame}

\frametitle{Prostori višjih dimenzij}
Posplošitve pojmov in rezultatov.

\end{frame}
%%%%%%%%%%%%%%%%%%%%%%%%%%%%%%%%%%%%%%%%
\begin {frame}

\frametitle{Včrtani kvadrat}
Vprašanje včrtanega kvadrata še nerešeno.

\end{frame}
%%%%%%%%%%%%%%%%%%%%%%%%%%%%%%%%%%%%%%%%
\begin {frame}

\frametitle{Skica}
\href{https://www.geogebra.org/calculator/rjr5wbun}{Skica prvega dokaza} 

\end{frame}
%%%%%%%%%%%%%%%%%%%%%%%%%%%%%%%%%%%%%%%%
\end{document}
